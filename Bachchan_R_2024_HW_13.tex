\documentclass[12pt, a4paper]{article}
\pagestyle{empty}
\usepackage[margin = 1in]{geometry}
\usepackage{mathtools, amssymb}
\usepackage{newtxtext, newtxmath}
\usepackage{enumerate}
\usepackage{setspace}
\onehalfspacing

\begin{document}
	\begin{center}
		{\Large PHYS 20323/60323: Fall 2024 - LaTeX Example}
	\end{center}
	
	\vspace{0.5cm}
	
	\begin{enumerate}
		\item An electron is found to be in the spin state (in the $z$-basis): $\chi = A\begin{pmatrix} 3i \\ 4 \end{pmatrix}$
		\begin{enumerate}[(a)]
			\item (5 points) Determine the possible values of $A$ such that the state is normalized.
			\item (5 points) Find the expectation values of the operators ${\color{red}S_x}$, $\color{purple}{S_y}$, $\color{orange}{S_z}$ and $\vec{S}^2$.
		\end{enumerate}
		The matrix representations in the $z$-basis for the components of electron spin operators are given by:
		\begin{equation*}
			{\color{red}\textbf{S}_x = \frac{\hbar}{2}
			\begin{pmatrix}
				0 & 1 \\
				1 & 0
			\end{pmatrix}}\,{\color{purple}; \qquad 
			\textbf{S}_y = \frac{\hbar}{2}
			\begin{pmatrix}
				0 & -i \\
				i & 0
			\end{pmatrix}}\,{\color{orange}; \qquad
			\textbf{S}_z = \frac{\hbar}{2}
			\begin{pmatrix}
				1 & 0 \\
				0 & -1
			\end{pmatrix}}
		\end{equation*}
		\item The average electrostatic field in the earth’s atmosphere in fair weather is approximately given:
		\begin{equation}
			\vec{E} = E_0\left(Ae^{-\alpha z} + Be^{-\beta z}\right)\hat{z},
		\end{equation}
		where $A$, $B$, $\alpha$, $\beta$ are positive constants and $z$ is the height above the (locally flat) earth surface.
		\begin{enumerate}[(a)]
			\item (5 points) Find the average charge density in the atmosphere as a function of height
			\item (5 points) Find the electric potential as a function height above the earth.
		\end{enumerate}
		\item \textbf{The following questions refer to stars in the Table below.}\\
		Note: There may be multiple answers.\\
		\begin{tabular}{|c|c|c|c|c|c|}
			\hline
			       Name        &     Mass      &    Luminosity    &          Lifetime          & Temperature &   Radius    \\ \hline
			   $\beta$ Cyg.    & 1.3 $M_\odot$ &  3.5 $L_\odot$   &                            &             &             \\ \hline
			  $\alpha$ Cen.    & 1.0 $M_\odot$ &                  &                            &             & 1 $R_\odot$ \\ \hline
			   $\eta$ Car.     & 60. $M_\odot$ & $10^6$ $L_\odot$ &  $8.0 \times 10^5$ years   &             &             \\ \hline
			$\varepsilon$ Eri. & 6.0 $M_\odot$ & $10^3$ $L_\odot$ &                            &  20,000 K   &             \\ \hline
			  $\delta$ Scu.    & 2.0 $M_\odot$ &                  &  $5.0 \times 10^8$ years   &             & 2 $R_\odot$ \\ \hline
			  $\gamma$ Del.    & 0.7 $M_\odot$ &                  & $4.5 \times 10^{10}$ years &   5000 K    &             \\ \hline
		\end{tabular}
		\begin{enumerate}[(a)]
			\item (4 points) Which of these stars will produce a planetary nebula.
			\item (4 points) Elements heavier than \textit{Carbon} will be produced in which stars.
		\end{enumerate}
	\end{enumerate}
\end{document}













